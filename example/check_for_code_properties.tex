% Options for packages loaded elsewhere
\PassOptionsToPackage{unicode}{hyperref}
\PassOptionsToPackage{hyphens}{url}
%
\documentclass[
]{article}
\usepackage{lmodern}
\usepackage{amssymb,amsmath}
\usepackage{ifxetex,ifluatex}
\ifnum 0\ifxetex 1\fi\ifluatex 1\fi=0 % if pdftex
  \usepackage[T1]{fontenc}
  \usepackage[utf8]{inputenc}
  \usepackage{textcomp} % provide euro and other symbols
\else % if luatex or xetex
  \usepackage{unicode-math}
  \defaultfontfeatures{Scale=MatchLowercase}
  \defaultfontfeatures[\rmfamily]{Ligatures=TeX,Scale=1}
\fi
% Use upquote if available, for straight quotes in verbatim environments
\IfFileExists{upquote.sty}{\usepackage{upquote}}{}
\IfFileExists{microtype.sty}{% use microtype if available
  \usepackage[]{microtype}
  \UseMicrotypeSet[protrusion]{basicmath} % disable protrusion for tt fonts
}{}
\makeatletter
\@ifundefined{KOMAClassName}{% if non-KOMA class
  \IfFileExists{parskip.sty}{%
    \usepackage{parskip}
  }{% else
    \setlength{\parindent}{0pt}
    \setlength{\parskip}{6pt plus 2pt minus 1pt}}
}{% if KOMA class
  \KOMAoptions{parskip=half}}
\makeatother
\usepackage{xcolor}
\IfFileExists{xurl.sty}{\usepackage{xurl}}{} % add URL line breaks if available
\IfFileExists{bookmark.sty}{\usepackage{bookmark}}{\usepackage{hyperref}}
\hypersetup{
  pdftitle={Testing for circularity},
  hidelinks,
  pdfcreator={LaTeX via pandoc}}
\urlstyle{same} % disable monospaced font for URLs
\usepackage[margin=1in]{geometry}
\usepackage{color}
\usepackage{fancyvrb}
\newcommand{\VerbBar}{|}
\newcommand{\VERB}{\Verb[commandchars=\\\{\}]}
\DefineVerbatimEnvironment{Highlighting}{Verbatim}{commandchars=\\\{\}}
% Add ',fontsize=\small' for more characters per line
\usepackage{framed}
\definecolor{shadecolor}{RGB}{248,248,248}
\newenvironment{Shaded}{\begin{snugshade}}{\end{snugshade}}
\newcommand{\AlertTok}[1]{\textcolor[rgb]{0.94,0.16,0.16}{#1}}
\newcommand{\AnnotationTok}[1]{\textcolor[rgb]{0.56,0.35,0.01}{\textbf{\textit{#1}}}}
\newcommand{\AttributeTok}[1]{\textcolor[rgb]{0.77,0.63,0.00}{#1}}
\newcommand{\BaseNTok}[1]{\textcolor[rgb]{0.00,0.00,0.81}{#1}}
\newcommand{\BuiltInTok}[1]{#1}
\newcommand{\CharTok}[1]{\textcolor[rgb]{0.31,0.60,0.02}{#1}}
\newcommand{\CommentTok}[1]{\textcolor[rgb]{0.56,0.35,0.01}{\textit{#1}}}
\newcommand{\CommentVarTok}[1]{\textcolor[rgb]{0.56,0.35,0.01}{\textbf{\textit{#1}}}}
\newcommand{\ConstantTok}[1]{\textcolor[rgb]{0.00,0.00,0.00}{#1}}
\newcommand{\ControlFlowTok}[1]{\textcolor[rgb]{0.13,0.29,0.53}{\textbf{#1}}}
\newcommand{\DataTypeTok}[1]{\textcolor[rgb]{0.13,0.29,0.53}{#1}}
\newcommand{\DecValTok}[1]{\textcolor[rgb]{0.00,0.00,0.81}{#1}}
\newcommand{\DocumentationTok}[1]{\textcolor[rgb]{0.56,0.35,0.01}{\textbf{\textit{#1}}}}
\newcommand{\ErrorTok}[1]{\textcolor[rgb]{0.64,0.00,0.00}{\textbf{#1}}}
\newcommand{\ExtensionTok}[1]{#1}
\newcommand{\FloatTok}[1]{\textcolor[rgb]{0.00,0.00,0.81}{#1}}
\newcommand{\FunctionTok}[1]{\textcolor[rgb]{0.00,0.00,0.00}{#1}}
\newcommand{\ImportTok}[1]{#1}
\newcommand{\InformationTok}[1]{\textcolor[rgb]{0.56,0.35,0.01}{\textbf{\textit{#1}}}}
\newcommand{\KeywordTok}[1]{\textcolor[rgb]{0.13,0.29,0.53}{\textbf{#1}}}
\newcommand{\NormalTok}[1]{#1}
\newcommand{\OperatorTok}[1]{\textcolor[rgb]{0.81,0.36,0.00}{\textbf{#1}}}
\newcommand{\OtherTok}[1]{\textcolor[rgb]{0.56,0.35,0.01}{#1}}
\newcommand{\PreprocessorTok}[1]{\textcolor[rgb]{0.56,0.35,0.01}{\textit{#1}}}
\newcommand{\RegionMarkerTok}[1]{#1}
\newcommand{\SpecialCharTok}[1]{\textcolor[rgb]{0.00,0.00,0.00}{#1}}
\newcommand{\SpecialStringTok}[1]{\textcolor[rgb]{0.31,0.60,0.02}{#1}}
\newcommand{\StringTok}[1]{\textcolor[rgb]{0.31,0.60,0.02}{#1}}
\newcommand{\VariableTok}[1]{\textcolor[rgb]{0.00,0.00,0.00}{#1}}
\newcommand{\VerbatimStringTok}[1]{\textcolor[rgb]{0.31,0.60,0.02}{#1}}
\newcommand{\WarningTok}[1]{\textcolor[rgb]{0.56,0.35,0.01}{\textbf{\textit{#1}}}}
\usepackage{graphicx,grffile}
\makeatletter
\def\maxwidth{\ifdim\Gin@nat@width>\linewidth\linewidth\else\Gin@nat@width\fi}
\def\maxheight{\ifdim\Gin@nat@height>\textheight\textheight\else\Gin@nat@height\fi}
\makeatother
% Scale images if necessary, so that they will not overflow the page
% margins by default, and it is still possible to overwrite the defaults
% using explicit options in \includegraphics[width, height, ...]{}
\setkeys{Gin}{width=\maxwidth,height=\maxheight,keepaspectratio}
% Set default figure placement to htbp
\makeatletter
\def\fps@figure{htbp}
\makeatother
\setlength{\emergencystretch}{3em} % prevent overfull lines
\providecommand{\tightlist}{%
  \setlength{\itemsep}{0pt}\setlength{\parskip}{0pt}}
\setcounter{secnumdepth}{-\maxdimen} % remove section numbering

\title{Testing for circularity}
\author{}
\date{\vspace{-2.5em}}

\begin{document}
\maketitle

\hypertarget{prerequisites}{%
\section{Prerequisites}\label{prerequisites}}

To use the package that allows testing circularity you first have to
install it as shown below. To run the code in the box below you can just
press the green play button in the left top corner of the code box. You
might be asked in the console below this window to update the included
packages. Afterwards it might take a while to install all 3 packages.

\begin{Shaded}
\begin{Highlighting}[]
\KeywordTok{install.packages}\NormalTok{(}\StringTok{"Rcpp"}\NormalTok{, }\DataTypeTok{repos =} \StringTok{"https://packages.othr.de/cran/"}\NormalTok{ )}
\end{Highlighting}
\end{Shaded}

\begin{verbatim}
## Error in install.packages : Updating loaded packages
\end{verbatim}

\begin{Shaded}
\begin{Highlighting}[]
\KeywordTok{install.packages}\NormalTok{(}\StringTok{"devtools"}\NormalTok{, }\DataTypeTok{repos =} \StringTok{"https://packages.othr.de/cran/"}\NormalTok{)}
\end{Highlighting}
\end{Shaded}

\begin{verbatim}
## package 'devtools' successfully unpacked and MD5 sums checked
## 
## The downloaded binary packages are in
##  C:\Users\Simon\AppData\Local\Temp\RtmpA7uLlQ\downloaded_packages
\end{verbatim}

\begin{Shaded}
\begin{Highlighting}[]
\NormalTok{devtools}\OperatorTok{::}\KeywordTok{install_github}\NormalTok{(}\StringTok{"StarmanMartin/GCATR"}\NormalTok{)}
\end{Highlighting}
\end{Shaded}

\begin{verbatim}
## Skipping install of 'GCATR' from a github remote, the SHA1 (8afaff3a) has not changed since last install.
##   Use `force = TRUE` to force installation
\end{verbatim}

\hypertarget{self-complementary}{%
\section{Testing for self-complementarity}\label{self-complementary}}

Now that you have installed both packages you can use the GCATR package.

To execute the following code snippet you just have to press the little
green arrow on the right side of the code box. You should be shown FALSE
when executing the code snippet as the provided code is not
self-complementary.

\begin{Shaded}
\begin{Highlighting}[]
\KeywordTok{library}\NormalTok{(Rcpp)}
\end{Highlighting}
\end{Shaded}

\begin{verbatim}
## Warning: package 'Rcpp' was built under R version 3.5.3
\end{verbatim}

\begin{Shaded}
\begin{Highlighting}[]
\KeywordTok{library}\NormalTok{(GCATR)}
\end{Highlighting}
\end{Shaded}

\begin{verbatim}
## Loading required package: igraph
\end{verbatim}

\begin{verbatim}
## Warning: package 'igraph' was built under R version 3.5.3
\end{verbatim}

\begin{verbatim}
## 
## Attaching package: 'igraph'
\end{verbatim}

\begin{verbatim}
## The following objects are masked from 'package:stats':
## 
##     decompose, spectrum
\end{verbatim}

\begin{verbatim}
## The following object is masked from 'package:base':
## 
##     union
\end{verbatim}

\begin{verbatim}
## Loading required package: gridExtra
\end{verbatim}

\begin{verbatim}
## Warning: package 'gridExtra' was built under R version 3.5.3
\end{verbatim}

\begin{verbatim}
## Loading required package: rmarkdown
\end{verbatim}

\begin{verbatim}
## Warning: package 'rmarkdown' was built under R version 3.5.3
\end{verbatim}

\begin{verbatim}
## Loading required package: xtable
\end{verbatim}

\begin{verbatim}
## Warning: package 'xtable' was built under R version 3.5.3
\end{verbatim}

\begin{verbatim}
## Loading required package: knitr
\end{verbatim}

\begin{verbatim}
## Warning: package 'knitr' was built under R version 3.5.3
\end{verbatim}

\begin{Shaded}
\begin{Highlighting}[]
\KeywordTok{code_check_if_self_complementary}\NormalTok{(}\StringTok{'UGG GUG'}\NormalTok{)}
\end{Highlighting}
\end{Shaded}

\begin{verbatim}
## [1] FALSE
\end{verbatim}

There is three options to provide the string that represents the code
you are trying to test.

Option 1 is providing the string with a specific tuple length like this:

\begin{Shaded}
\begin{Highlighting}[]
\KeywordTok{library}\NormalTok{(Rcpp)}
\KeywordTok{library}\NormalTok{(GCATR)}
\KeywordTok{code_check_if_self_complementary}\NormalTok{(}\StringTok{'UGGGUG'}\NormalTok{, }\DecValTok{3}\NormalTok{)}
\end{Highlighting}
\end{Shaded}

\begin{verbatim}
## [1] FALSE
\end{verbatim}

Option 2 is providing the string split in tuples by spaces like so:

\begin{Shaded}
\begin{Highlighting}[]
\KeywordTok{library}\NormalTok{(Rcpp)}
\KeywordTok{library}\NormalTok{(GCATR)}
\KeywordTok{code_check_if_self_complementary}\NormalTok{(}\StringTok{'UGG GUG'}\NormalTok{)}
\end{Highlighting}
\end{Shaded}

\begin{verbatim}
## [1] FALSE
\end{verbatim}

The last option is to provide each tuple separate in a vector as so:
This last example should come up TRUE, as the provided code is ciruclar.

\begin{Shaded}
\begin{Highlighting}[]
\KeywordTok{library}\NormalTok{(Rcpp)}
\KeywordTok{library}\NormalTok{(GCATR)}
\KeywordTok{code_check_if_self_complementary}\NormalTok{(}\KeywordTok{c}\NormalTok{(}\StringTok{'UGG'}\NormalTok{, }\StringTok{'GUG'}\NormalTok{, }\StringTok{'GAG'}\NormalTok{))}
\end{Highlighting}
\end{Shaded}

\begin{verbatim}
## [1] FALSE
\end{verbatim}

\hypertarget{testing-for-maximum-circularity}{%
\section{Testing for maximum
circularity}\label{testing-for-maximum-circularity}}

To test for maximum circularity you might just execute the following
code snippet. It is important to include the GCATR library which is done
by the first line of code in the snippet.

Obviously the string can be provided in the different ways mentioned in
the testing for
\protect\hyperlink{self-complementary}{self-complementarity} section.

You should be shown FALSE when executing this code snippet as the
provided code is not maximum circular.

\begin{Shaded}
\begin{Highlighting}[]
\KeywordTok{library}\NormalTok{(Rcpp)}
\KeywordTok{library}\NormalTok{(GCATR)}
\KeywordTok{code_check_if_circular}\NormalTok{(}\StringTok{'UGG GUG'}\NormalTok{)}
\end{Highlighting}
\end{Shaded}

\begin{verbatim}
## [1] FALSE
\end{verbatim}

\hypertarget{testing-for-k-circularity}{%
\section{Testing for k-circularity}\label{testing-for-k-circularity}}

Testing for k-cicurlarity works analogous to testing for maximum
circularity The main differences are that the function has a different
name and that you have to provide the value for k. Obviously the string
can be provided in the different ways mentioned in the testing for
\protect\hyperlink{self-complementary}{self-complementarity} section.

k = 1, should come up FALSE

\begin{Shaded}
\begin{Highlighting}[]
\KeywordTok{library}\NormalTok{(Rcpp)}
\KeywordTok{library}\NormalTok{(GCATR)}
\KeywordTok{code_check_if_k_circular}\NormalTok{(}\DecValTok{1}\NormalTok{,}\StringTok{'UGGGUG'}\NormalTok{, }\DecValTok{3}\NormalTok{)}
\end{Highlighting}
\end{Shaded}

\begin{verbatim}
## [1] TRUE
\end{verbatim}

k = 2, should come up TRUE

\begin{Shaded}
\begin{Highlighting}[]
\KeywordTok{library}\NormalTok{(Rcpp)}
\KeywordTok{library}\NormalTok{(GCATR)}
\KeywordTok{code_check_if_k_circular}\NormalTok{(}\DecValTok{2}\NormalTok{,}\StringTok{'ACGGUACGUCGGUAC'}\NormalTok{,}\DecValTok{3}\NormalTok{)}
\end{Highlighting}
\end{Shaded}

\begin{verbatim}
## [1] FALSE
\end{verbatim}

k = 3, should come up false

\begin{Shaded}
\begin{Highlighting}[]
\KeywordTok{library}\NormalTok{(Rcpp)}
\KeywordTok{library}\NormalTok{(GCATR)}
\KeywordTok{code_check_if_k_circular}\NormalTok{(}\DecValTok{3}\NormalTok{,}\StringTok{'ACG GUA CGU CGG UAC'}\NormalTok{)}
\end{Highlighting}
\end{Shaded}

\begin{verbatim}
## [1] TRUE
\end{verbatim}

k = 4, should come up true

\begin{Shaded}
\begin{Highlighting}[]
\KeywordTok{library}\NormalTok{(Rcpp)}
\KeywordTok{library}\NormalTok{(GCATR)}
\KeywordTok{code_check_if_k_circular}\NormalTok{(}\DecValTok{4}\NormalTok{,}\KeywordTok{c}\NormalTok{(}\StringTok{'GGU'}\NormalTok{, }\StringTok{'GGC'}\NormalTok{, }\StringTok{'ACU'}\NormalTok{, }\StringTok{'ACC'}\NormalTok{, }\StringTok{'AGC'}\NormalTok{, }\StringTok{'AGU'}\NormalTok{, }\StringTok{'GAC'}\NormalTok{, }\StringTok{'GAU'}\NormalTok{, }\StringTok{'GUC'}\NormalTok{, }\StringTok{'GUU'}\NormalTok{, }\StringTok{'AAU'}\NormalTok{, }\StringTok{'AUU'}\NormalTok{, }\StringTok{'AAC'}\NormalTok{, }\StringTok{'AUC'}\NormalTok{, }\StringTok{'GCU'}\NormalTok{, }\StringTok{'GCC'}\NormalTok{))}
\end{Highlighting}
\end{Shaded}

\begin{verbatim}
## [1] FALSE
\end{verbatim}

\hypertarget{testing-for-cn-circularity}{%
\section{Testing for cn-circularity}\label{testing-for-cn-circularity}}

Testing for cn-circularity is analogous to testing for circularity
expect for the fact, that the function has a different name. Obviously
the string can be provided in the different ways mentioned in the
testing for \protect\hyperlink{self-complementary}{self-complementarity}
section.

This example should come up false.

\begin{Shaded}
\begin{Highlighting}[]
\KeywordTok{library}\NormalTok{(Rcpp)}
\KeywordTok{library}\NormalTok{(GCATR)}
\KeywordTok{code_check_if_cn_circular}\NormalTok{(}\StringTok{'UGG GUG'}\NormalTok{)}
\end{Highlighting}
\end{Shaded}

\begin{verbatim}
## [1] FALSE
\end{verbatim}

While this one should come up true.

\begin{Shaded}
\begin{Highlighting}[]
\KeywordTok{library}\NormalTok{(Rcpp)}
\KeywordTok{library}\NormalTok{(GCATR)}
\KeywordTok{code_check_if_cn_circular}\NormalTok{(}\KeywordTok{c}\NormalTok{(}\StringTok{'GGU'}\NormalTok{,}\StringTok{'GGC'}\NormalTok{,}\StringTok{'ACU'}\NormalTok{,}\StringTok{'ACC'}\NormalTok{,}\StringTok{'AGC'}\NormalTok{,}\StringTok{'AGU'}\NormalTok{,}\StringTok{'GAC'}\NormalTok{,}\StringTok{'GAU'}\NormalTok{,}\StringTok{'GUC'}\NormalTok{,}\StringTok{'GUU'}\NormalTok{,}\StringTok{'AAU'}\NormalTok{,}\StringTok{'AUU'}\NormalTok{,}\StringTok{'AAC'}\NormalTok{,}\StringTok{'AUC'}\NormalTok{,}\StringTok{'GCU'}\NormalTok{,}\StringTok{'GCC'}\NormalTok{))}
\end{Highlighting}
\end{Shaded}

\begin{verbatim}
## [1] TRUE
\end{verbatim}

\hypertarget{testing-for-comma-freeness}{%
\section{Testing for comma-freeness}\label{testing-for-comma-freeness}}

Testing for comma-freeness also works analogous to the other function
calls. Just execute the following code snippet to try it out. The
example should come up true. Obviously the string can be provided in the
different ways mentioned in the testing for
\protect\hyperlink{self-complementary}{self-complementarity} section.

\begin{Shaded}
\begin{Highlighting}[]
\KeywordTok{library}\NormalTok{(Rcpp)}
\KeywordTok{library}\NormalTok{(GCATR)}
\KeywordTok{code_check_if_comma_free}\NormalTok{(}\StringTok{'GGU GGC ACU ACC AGC AGU GAC GAU GUC GUU AAU AUU AAC AUC GCU GCC'}\NormalTok{)}
\end{Highlighting}
\end{Shaded}

\begin{verbatim}
## [1] TRUE
\end{verbatim}

\end{document}
